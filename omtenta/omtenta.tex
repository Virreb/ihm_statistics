\documentclass[a4paper,10pt]{article}
\usepackage[utf8]{inputenc}
\usepackage{color}
\usepackage{graphicx}
\usepackage{subfigure}
\usepackage{float}
\usepackage{geometry}
\usepackage{setspace}
\usepackage{amsmath}

\title{Omtentamen\\ {Statistik HT18 (DMS17H)}}
\author{Victor Bäckman, victor.backman@advectas.se}

\begin{document}
\maketitle
\vspace{2cm}
Namn:
{\rule{13cm}{0.4pt}}
\vspace{5cm}
\begin{center}
  Tillåtna hjälpmedel: miniräknare och nytt formelblad.\\
    \vspace{3cm}
  {\huge Lycka till!}
\end{center}

\newpage
\begin{enumerate}
    \item{En pop-up butik har $6$ stycken olika produkter med id: $\{1, 2, 3, 4, 5, 6\}$. Produkterna kan delas upp i två olika produktgrupper; $A=[1, 3]$ och $B=[2, 4, 5, 6]$.}
    \begin{enumerate}
        \item{Skissa upp produktgrupperna i ett Venndiagram, är de disjunkta?}
        \vspace{8cm}
        \item{Du försöker locka kunder genom att anordna en klassisk fiskedamm i butiken. Du lägger en av varje produkt i en låda och drar slumpmässigt ut en produkt per fiskare. Hur stor är sannolikheten att den första fiskaren får en av produkterna i grupp $B$?}
        \newpage
        \item{Vad är sannolikheten för nästa fiskare att också få en produkt från grupp $B$? (dammen fylls inte på med nya produkter)}
    \end{enumerate}
    \newpage
    \item{Hur många olika kombinationer med $3$ stycken produkter kan du skapa utifrån de $6$ ovan? Ordningen mellan produkterna spelar ingen roll och en produkt får endast finnas med en gång per kombination.}
    \newpage
    \item{Pop-up butiken lockar en hel del folk, men alla som kommer in i butiken köper inte. Du blir nyfiken på hur stor andel det är som faktisk köper och får låna en utrustning som mäter antalet kunder som går in i butiken per timme och väger det mot antalet köp. Eftersom du bara har lånat mätanordningen behöver du lämna tillbaka den efter 4 timmar, resultatet kan ses i tabellen ovan.
    \begin{table}
        \centering
        \begin{tabular}[H]{c|c|c}
        \textbf{Timme} & \textbf{Antal besökare $(x)$}  & \textbf{Antal köp $(y)$}\\ \hline
        13:00                    & 61                   & 13                           \\
        14:00                    & 47                   & 7                           \\
        15:00                    & 23                   & 4                           \\
        16:00                    & 83                   & 15
        \end{tabular}
    \end{table}
  }
    \begin{enumerate}
        \item{Skissa upp datan i en "scatter-plot". Ser du någon korrelation?}
        \newpage
        \item{Bevisa att det finns en korrelation genom att beräkna korrelationskoefficienten $r$}
        \newpage
        \item{Beskriv korrelationen med en linjär regression ($\hat{y}=kx+m$). Skissa även upp linjen i din "scatter-plot".}
        \newpage
        \item{Använd din linjära regression för att uppskatta hur många köp butiken hade haft om det var $150$ besökare under en timme. Kan du lita på värdet?}
        \newpage
        \item{Tolka värdet för $k$ och $m$, vad betyder dem i detta fallet? Är dem rimliga? Beskriv med ord!}
    \end{enumerate}
    \newpage
    \item{Pop-up butiken mäter kundnöjdhet genom att kunderna vid utgången får betygssätta besöket genom olika "smiley-knappar". Du uppskattar att under en hel dag har pop-up butiken haft ungefär $400$ besökare varav $22$ har markerat att de var missnöjda. Bilda ett $90\%$ konfidensintervall för detta.}
\end{enumerate}

\end{document}
