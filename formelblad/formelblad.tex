\documentclass[a4paper,10pt]{article}
\usepackage[utf8]{inputenc}
\usepackage{color}
\usepackage{graphicx}
\usepackage{subfigure}
\usepackage{float}
\usepackage{geometry}
\usepackage{setspace}
\usepackage{amsmath}
\usepackage{pdfpages}

\title{Formelblad \\ {\large Statistik HT18 (DMS17H)}}
\author{Victor Bäckman, victor.backman@advectas.se}
\begin{document}
\maketitle

\subsection*{Operationer}
Fakultet\\
$0! = 1$ av definition \\
$1! = 1$ \\
$2! = 1 \cdot 2 = 2$ \\
$3! = 1 \cdot 2 \cdot 3 = 6$ \\
$4! = 1 \cdot 2 \cdot 3 \cdot 4 = 3! \cdot 4 = 24$ \\
och så vidare.\\
\vspace{1mm} \\
Binomialkoefficient: $\binom{n}{k} = \frac{n!}{k!(n-k)!}$

\subsection*{Deskriptiv statistik}
Medelvärde: $\bar{x} = \frac{\sum_{i=1}^n x_i}{n}$ \\
Standardavvikelse: $\sigma = \sqrt{\frac{\sum_{i=1}^n (x_i - \bar{x})^2}{n}}$ \\
Varians: $Var(X) = \sigma^2$ \\
Medianens position i en lista $= \frac{n+1}{2}$ \\
Kvartilavstånd $=q_3 -q_1$

\subsection*{Korrelation}
Korrelationskoefficient: $r=\frac{\sum_{i=1}^n (x_i - \bar{x})(y_i - \bar{y})}{\sqrt{\sum_{i=1}^n (x_i - \bar{x})^2\sum_{i=1}^n (y_i - \bar{y})^2}}$ \\
\, \\
Räta linjens ekvation: $y = kx + m$
\begin{enumerate}
  \item{Grafiskt genom att välja ut två punkter: \\ \, \\ $k=\frac{\Delta y}{\Delta x} = \frac{y_2 - y_1}{x_2 - x_1}$\\ \, \\ $m=y_1 - kx_1$}
  \item{Genom regression: \\ \, \\ $k=\frac{\sum_{i=1}^n (x_i - \bar{x})(y_i - \bar{y})}{\sum_{i=1}^n (x_i - \bar{x})^2}$ \\
\, \\
$m=\bar{y} - k\bar{x}$
  }
\end{enumerate}
\newpage

\subsection*{Sannolikhetslära}
$S$: utfallsrum, unika element \\
$A,B \in S$ (A,B är händelser eller delmängder av S \\
$A,B$ är disjunkta om de inte har några gemensamma element (inget överlapp) \\
$A,B$ är beroende om de påverkar förutsättningarna för varandra \\
$A \cup B$: Unionen mellan A och B (A eller B) \\
$A \cap B$: Snittet mellan A och B (A och B) \\
Sannolikheten för A $= P_r(A) = \frac{\text{antal gynnsamma utfall}}{\text{antal utfall}}$ \\
 \\
Regler:
\begin{enumerate}
  \item{$0 \leq P_r(A) \leq 1$}
  \item{$\sum_{i=1}^n P_r(A_i) = 1$}
  \item{$P_r(A^c) = 1 - P_r(A)$}
\end{enumerate}
{\bf Additionssatsen:} \\
A,B disjunkta (inget överlapp): $P_r(A \cup B) = P_r(A) + P_r(B)$ \\
A,B ej disjunkta (överlapp): $P_r(A \cup B) = P_r(A) + P_r(B) - P_r(A \cap B)$ \\
 \vspace{1mm} \\
{\bf Multiplikationssatsen:} \\
A,B oberoende: $P_r(A \cap B) = P_r(A) \cdot P_r(B)$ \\
A,B beroende: $P_r(A \cap B) = P_r(B) \cdot P_r(A|B) = P_r(A) \cdot P_r(B|A) = P_r(B \cap A)$ \\
 \vspace{1mm} \\
{\bf Betingad sannolikhet:} \\
Sannolikheten att A händer om B redan har hänt: $P_r(A|B) = \frac{P_r(A \cap B)}{P_r(B)}$ \\
A och B är oberoende om: $P_r(A|B) = P_r(A)$ \\
\vspace{1mm} \\
Odds för händelse A: $O(A) = \frac{P_r(A)}{P_r(A^c)} = \frac{P_r(A)}{1-P_r(A)}$
\vspace{1cm} \\
\subsection*{Kombinatorik}
Kombinationer utan återinläggning: $C_n^k = \binom{n}{k} = \frac{n!}{k!(n-k)!}$ \\
Kombinationer med återinläggning: $C_n^k = \binom{n+k-1}{k} = \frac{(n+k-1)!}{k!(n-1)!}$ \\
Permutationer utan återinläggning: $P_n^k = \frac{n!}{(n-k)!}$ \\
Permutationer med återinläggning: $P_n^k = n^k$ \\
$n: $ totala antalet element, $k: $ antalet uttagna element

\newpage

\subsection*{Sannolikhetsfördelningar}
Väntevärde: $\mu = \sum_{i=1}^g P_r(x_i) x_i$, \, $g=$antal enheter \\
Standardavvikelse: $\sigma = \sqrt{\sum_{i=1}^g P_r(x_i) (x_i - \mu)^2}$ \\
\vspace{1mm} \\
{\bf Normalfördelning:}
\begin{enumerate}
    \item{Är symmetrisk}
    \item{Har kontinuerliga värden}
\end{enumerate}
$\Rightarrow f(x)=\frac{1}{\sigma\sqrt{2\pi}}e^{\frac{-1}{2}(\frac{x-\mu}{\sigma})^2}$ \\
\vspace{1mm}\\
Transformering för att läsa av tabell (längst bak): $z = \frac{x-\mu}{\sigma}$ \\
\vspace{1mm} \\
{\bf Binomialfördelning:}
\begin{enumerate}
    \item{Har diskreta värden}
    \item{Alla försök ska vara oberoende av varandra}
    \item{Utfallet ska vara binärt (True eller False)}
\end{enumerate}
$\Rightarrow P_r(k) = \binom{n}{k} P^k(1-P)^{n-k}$, \\
\, \\
$n=$ antal försök, $P=$ sannolikhet att ett försök lyckas. \\
\, \\
Nyckeltal: $\mu=nP$,\, $\sigma=\sqrt{nP(1-P)}$ \\
Approximera binomialfördelning med en normalfördelning om: $nP(1-P) > 5$
\vspace{1cm} \\
\subsection*{Konfidensintervall}
Krav:
\begin{enumerate}
  \item{Elementen i stickprovet ska vara oberoende}
  \item{$nP(1-P) > 5$ (kan approximeras som en normalfördelning)}
\end{enumerate}
$\Rightarrow P \pm Z \cdot \sqrt{\frac{P(1-P)}{n}}$, \\
\, \\
$Z$ utläses ur tabell enligt $Pr(Z) = 1 - \frac{\alpha}{2}$, $n$ är antalet element och \\
signifikansnivån $\alpha = 1-$konfidensnivån



\includepdf[pages=-]{standardnormaltable.pdf}

\end{document}
