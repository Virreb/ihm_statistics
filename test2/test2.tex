\documentclass[a4paper,10pt]{article}
\usepackage[utf8]{inputenc}
\usepackage{color}
\usepackage{graphicx}
\usepackage{subfigure}
\usepackage{float}
\usepackage{geometry}
\usepackage{setspace}
\usepackage{amsmath}

\title{Kombinatorik och sannolikhetslära \\ {\large Test 2 - Statistik HT18 (DMS17H)}}
\author{Victor Bäckman}

\begin{document}
\maketitle
\vspace{2cm}
Namn:
{\rule{13cm}{0.4pt}}
\vspace{8cm}
\begin{center}
  {\huge Lycka till!}
\end{center}
\newpage
\begin{enumerate}
  \item{Du har precis startat en e-handel och har just nu fem produkter du säljer. På hemsidan vill du visa upp   två produkter som hela tiden byts ut automatiskt till två nya, samma produkt får endast visas en gång per annons. Det går alltså inte att ha två likadana produkter bredvid varandra.}
    \begin{enumerate}
      \item{Hur många olika uppsättningar finns det, om {\bf produkternas ordning inte spelar roll}?}
      \vspace{6cm}
      \item{Hur många olika uppsättningar finns det, om {\bf produkternas ordning spelar roll}?}
      \vspace{6cm}
      \item{En besökare stannar i snitt 120 sekunder på din sida. Hur ofta ska du byta produkt-par för att alla produkter ska hinnas visas upp? Ordningen mellan produkterna spelar ingen roll.}
    \end{enumerate}
    \newpage
  \item{En bilverkstad har specialiserat sig på att byta lampor på bilar. I snitt är $56\%$ av ärendena att endast byta framlampa, $34\%$ att byta både fram-och-baklampa och resterande att endast byta baklampa.}
    \begin{enumerate}
        \item{Hur stor andel byter endast baklampa?}
        \vspace{2cm}
        \item{Visualisera händelserna $A=$"Byter framlampa" och $B=$"Byter baklampa" i ett Venn-diagram.}
        \vspace{5cm}
        \item{Vad är sannolikheten att en kund byter baklampa om de redan har bytt framlampa?}
        \vspace{5cm}
        \item{Påverkas sannolikheten att byta baklampa, om man redan har bytt framlampa? Motivera!}
    \end{enumerate}
  \newpage
  \item{Du får in några första ordrar till din e-handel, grattis! Du har just nu 10st produkter totalt: $S=\{0, 1, 2, 3, 4, 5, 6, 7, 8, 9\}$. De två första köpen $K_1, K_2$ innehöll följande produkt-id: $K_1=\{1, 3, 4, 6\}$ och $K_2=\{2, 3, 5\}$.}
    \begin{enumerate}
      \item{Är köpen $K_1$ och $K_2$ disjunkta? Motivera!}
      \vspace{3cm}
      \item{Antag att du köpt på dig ett stort lager från Kina, påverkar $K_1$ och $K_2$ varandra (är de beroende eller oberoende)? Motivera med ord!}
      \vspace{3cm}
      \item{Vad är sannolikheten att köp $K_1$ {\bf eller} $K_2$ sker?}
      \vspace{6cm}
      \item{Vad är sannolikheten att köp $K_1$ {\bf och} $K_2$? sker}
    \end{enumerate}
\end{enumerate}
\vfill
\end{document}
