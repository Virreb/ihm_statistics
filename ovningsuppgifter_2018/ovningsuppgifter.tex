\documentclass[a4paper,10pt]{article}
\usepackage[utf8]{inputenc}
\usepackage{color}
\usepackage{graphicx}
\usepackage{subfigure}
\usepackage{float}
\usepackage{geometry}
\usepackage{setspace}
\usepackage{amsmath}

\title{Övningsuppgifter inför tentamen\\ Statistik}
\author{Victor Bäckman, victor.backman@advectas.se}

\begin{document}
\maketitle
\vspace{2cm}
Namn:
{\rule{13cm}{0.4pt}}
\vspace{5cm}
\begin{center}
  {\huge Lycka till!}
\end{center}

\newpage
\begin{enumerate}
    \item{{\bf Deskriptiv statistik} \\ Du har en lista över hur länge $5$ olika besökare stannade på din hemsida i sekunder: $A=[165, 188, 159, 170, 198]$. Beräkna:}
    \begin{enumerate}
      \item{Medelvärdet $\bar{A}$}
      \vspace{3cm}
      \item{Medianen $m$}
      \vspace{3cm}
      \item{Standardavvikelsen $\sigma$}
      \vspace{3cm}
      \item{Variansen $var(x)$}
    \end{enumerate}
    \newpage
    \item{{\bf Sannolikhetsfördelningar} \\ Du har en normalfördelning över försäljning i tusentals kronor med $\mu = 8$ och $\sigma=3$.}
    \begin{enumerate}
      \item{Beräkna och tolka $P_r(x \leq 2)$}
      \vspace{6cm}
      \item{Beräkna och tolka $P_r(x \geq 6)$}
      \vspace{6cm}
      \item{Beräkna och tolka $P_r(4 \leq x \leq 5)$}
      \newpage
      \item{Beräkna $a$ för $P_r(x \leq a) = 35\%$ och tolka svaret}
    \end{enumerate}
    \newpage
    \item{{\bf Sannolikhetsfördelningar}\\ Du gillar att käka ägg till kvällsmat och vet från historisk data att $20\%$ blir himmelskt perfekta. Antag att äggens tillagning inte påverkar varandra.}
    \begin{enumerate}\label{egg}
        \item{Vad är sannolikheten att $3$ ägg blir perfekta, om du steker $6$ stycken?}
        \vspace{15cm}
        \item{Vad är sannolikheten att $3$ ägg inte blir perfekta, om du steker $6$ stycken?}
        \newpage
        \item{Du tar ett extrajobb på restaurang, vad är sannolikheten att fler än $20$ ägg blir perfekta, om du steker $100$ stycken?}
        \vspace{15cm}
        \item{Vad är det väntade antalet perfekta ägg om du stekar $100$ stycken?}
    \end{enumerate}
    \newpage
    \item{{\bf Konfidensintervall}\\Beräkna konfidensintervall för sannolikheten i uppgift \ref{egg}c om konfidensnivån är:}
    \begin{enumerate}
        \item{$90\%$}
        \vspace{6cm}
        \item{$50\%$}
        \vspace{6cm}
        \item{$10\%$}
    \end{enumerate}
    \newpage
    \item{{\bf Kombinatorik} \\ Du har $7$ stycken olika produkter och vill göra erbjudanden om $3$st. Beräkna antalet möjliga uppsättningar om:}
    \begin{enumerate}
        \item{du endast vill ha samma produkt en gång per erbjudande och ordningen inte spelar roll}
        \vspace{4cm}
        \item{ordningen inte spelar roll och samma produkt får vara med flera gånger per erbjudande}
        \vspace{4cm}
        \item{du endast vill ha samma produkt en gång per erbjudande och ordningen spelar roll}
        \vspace{4cm}
        \item{ordningen spelar roll och samma produkt får vara med flera gånger per erbjudande}
    \end{enumerate}
    \newpage
    \item{{\bf Korrelation} \\
    \begin{table}
        \centering
        \begin{tabular}[H]{c|c}
        \textbf{$(x)$}  & \textbf{$(y)$}\\ \hline
        1                        & 9                           \\
        3                        & 5                           \\
        5                       & 0                           \\
        8                       & -7
        \end{tabular}
    \end{table}
  }
    \begin{enumerate}
        \item{Skissa upp datan i en "scatter-plot". Ser du någon korrelation?}
        \newpage
        \item{Beräkna korrelationskoefficienten $r$}
        \newpage
        \item{Beskriv korrelationen med en linjär regression ($\hat{y}=kx+m$).}
    \end{enumerate}
\newpage
    \item{{\bf Sannolikhetslära} \\ $S=\{a, b, c, d, e, f\}$, $A=[a, b, c]$, $B=[c, d, e, f]$. Antag $A, B$ oberoende.}
    \begin{enumerate}
        \item{Skissa upp $A, B$ i ett Venndiagram, är de disjunkta?}
        \vspace{6cm}
        \item{Beräkna $A^c)$}
        \vspace{3cm}
        \item{Beräkna $P_r("Andel av S i A" \cap "Andel av S i B")$}
        \vspace{3cm}
        \item{Beräkna sannolikheten att ett element finns i A eller B  $P_r(A \cup B)$}
        \vspace{3cm}
        \item{Beräkna oddset för att dra $a$ eller $b$ ur $S$}
    \end{enumerate}
    \newpage
    \item{{\bf Betingad sannolikhet}\\En femtedel av alla fästingar bär på borrelia. Risken att borrelia överförs till en människa vid ett bett är $60\%$ (påhittad data)}. \\
    \begin{enumerate}
        \item{Vad är sannolikheten att borrelia överförs till en människa vid bett?}
        \vspace{6cm}
        \item{Vad är sannolikheten att drabbas av borrelia, om man blir biten $3$ gånger?}
    \end{enumerate}
    \newpage
    \item{{\bf Stickprov} \\ Diskutera:}
    \begin{enumerate}
        \item{Varför gör man stickprov?}
        \vspace{5cm}
        \item{Vad finns det för möjliga felkällor när man gör ett stickprov?}
        \vspace{5cm}
        \item{Vad kan man göra för att minimera felet när man använder stickprov?}
        \vspace{5cm}
        \item{Definiera den centrala gränsvärdessatsen.}
    \end{enumerate}
\end{enumerate}

\end{document}
