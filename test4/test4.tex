\documentclass[a4paper,10pt]{article}
\usepackage[utf8]{inputenc}
\usepackage{color}
\usepackage{graphicx}
\usepackage{subfigure}
\usepackage{float}
\usepackage{geometry}
\usepackage{setspace}
\usepackage{amsmath}

\title{Korrelation \\ {\large Test 4}}
\author{Victor Bäckman}

\begin{document}
\maketitle
\vspace{2cm}
Namn:
{\rule{13cm}{0.4pt}}
\vspace{8cm}
\begin{center}
  {\huge Lycka till!}
\end{center}
\newpage
\begin{enumerate}
    \item{Din nystartade e-handel från förra testet går som sagt som tåget! Du har en teori om att antalet köp per dag beror på antalet besök på din hemsida. Under några dagar antecknar du antalet sessioner och antalet köp, resultatet kan ses i tabellen ovan.
      \begin{table}
        \centering
        \begin{tabular}[H]{r|l}
        \textbf{Hundratals sessioner} & \textbf{Antal köp} \\ \hline
        2                            & 4                  \\
        5                            & 11                  \\
        6                            & 10                 \\
        8                            & 19                 \\
        9                            & 21
        \end{tabular}
    \end{table}
   }
    \begin{enumerate}
        \item{Skissa upp resultatet i en "scatter-plot". Ser du någon korrelation?}
        \newpage
        \item{Bevisa att det finns en korrelation genom att beräkna korrelationskoefficienten $r$}
        \newpage
        \item{Beskriv korrelationen med en linjär regression ($\hat{y}=kx+m$). Skissa även upp linjen i din "scatter-plot"}
        \newpage
      \item{Tolka värdet för $k$ och $m$, vad betyder dem i detta fallet? Beskriv med ord! (Kom ihåg att det är hundratals sessioner i tabellen)}
    \end{enumerate}
  \newpage
  \item{Du har tidigare påvisat att om du skickar $200$ av dina paket med Postnord (i separata sändningar) så är det sannolikt att $5\%$ försvinner. Bilda ett $90\%$ konfidensintervall för detta.}
\end{enumerate}
\end{document}
