\documentclass[a4paper,10pt]{article}
\usepackage[utf8]{inputenc}
\usepackage{color}
\usepackage{graphicx}
\usepackage{subfigure}
\usepackage{float}
\usepackage{geometry}
\usepackage{setspace}
\usepackage{amsmath}

\title{Sannolikhetsfördelningar \\ {\large Test 3 - Statistik HT18 (DMS17H)}}
\author{Victor Bäckman}

\begin{document}
\maketitle
\vspace{2cm}
Namn:
{\rule{13cm}{0.4pt}}
\vspace{8cm}
\begin{center}
  {\huge Lycka till!}
\end{center}
\newpage
\begin{enumerate}
  \item{Din nystartade e-handel från förra testet går som tåget! Antalet produkter du säljer per dag följer en normalfördelning med $\mu = 10$ och $\sigma=5$.}
    \begin{enumerate}
      \item{Hur stor är sannolikheten att du säljer $4$ eller färre produkter?}
      \vspace{8cm}
      \item{Hur stor är sannolikheten att du säljer $8$ eller fler produkter?}
      \newpage
      \item{Hur stor är sannolikheten att du säljer mellan $1$ och $3$ produkter?}
      \vspace{6cm}
      \item{Hur många produkter måste du sälja per dag för att det ska vara fler än $25\%$ av försäljningen?}
    \end{enumerate}
    \newpage
  \item{Du anlitar Postnord för att skicka dina paket (R.I.P). Det finns uppmätt att $5\%$ av alla paket de hanterar kommer fel (påhittat). Du skickar alla paket i separata sändningar som därav kan anses oberoende.}
    \begin{enumerate}
        \item{Vad är sannolikheten att $2$ paket försvinner, om du skickar $8$ stycken?}
        \vspace{8cm}
        \item{Vad är sannolikheten att fler än $15$ paket försvinner, om du skickar $200$ stycken i separata sändingar?}
    \end{enumerate}
\end{enumerate}
\vfill
\end{document}
